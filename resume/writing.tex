%-------------------------------------------------------------------------------
%	SECTION TITLE
%-------------------------------------------------------------------------------
\cvsection{Projects}


%-------------------------------------------------------------------------------
%	CONTENT
%-------------------------------------------------------------------------------
\begin{cventries}
  \cventry
    {\hyperref[https://github.com/coderade]{https://github.com/coderade}} % Role
    {Open Source Projects}% Title
    {Github} % Location
    {Jan. 2014 - PRESENT} % Date(s)
    {
      \begin{cvitems} % Description(s)
       \item{Place where I like to share projects and technologies that I am currently studying (always focusing in documenting and following the best standards, making easy for anyone who want use the projects), besides contributing with some community projects }
      \end{cvitems}
    }
%---------------------------------------------------------
  \cventry
    {Proprietary software} % Role
    {Supervisory system}% Title
    {Rumo} % Location
    {Sep. 2017 - PRESENT} % Date(s)
    {
      \begin{cvitems} % Description(s)
       \item{Software Project with the goal of implementing and improving real-time communication systems between devices used by the locomotives/trains concessioned by Rumo and Internet of things (IOT) for handling data such as BI, monitoring, Machine Learning and etc.}
       \item{The initial focus of the project was the initiation or improvement of the equipment used in the route by the company's RUMO trains running an average of 5 million rules executed daily on the AWS for sending data to systems such as PRTG System (monitoring) and Oracle Analytics Cloud (BigData).}
       \item{This project was developed using the maximum of AWS Stack (EC2, DynamoDB, IOT, S3, Lambda, MQTT, SQS, SNS, Kinesis) for data processing and the Big Data stack (HDFS Hadoop, Cloudera, Impala, Hive, Oracle Analytics Cloud) for our Analytics system. In addition to using the most the open-source technologies such as Python, Node.Js, Javascript (ES6), Java, AngularJS and others.}
      \end{cvitems}
    }
%---------------------------------------------------------
  \cventry
     {\hyperref[https://portalzoom.rumolog.com]{https://portalzoom.rumolog.com}}
    {Rumo Zoom Portal }% Title
    {Rumo} % Location
    {Jul. 2019 - Set. 2019} % Date(s)
    {
      \begin{cvitems} % Description(s)
       \item{Web application designed to be used as an integration tool between Zoom and Rumo system for meeting scheduling.}
       \item{In this project I was be able to develop as a Full Stack Engineer, creating its architecture, developing since the Frontend to the backend, Working as SysOps and Managing its database.}
        \item{As I was the main responsible for this project, I tried to use the best technologies for the project needs and deadline.}
         \item{In this way, the project has a backend developed using Python (Flask) and frontend in ReactJS/Redux and Bootstrap, uses a NOSQL database with MongoDB, is served with Nginx and secured with a Let’s Encrypt SSL certificate. And everything is up using only a command using Docker/Docker Compose!}
      \end{cvitems}
    }

%---------------------------------------------------------
  \cventry
    {Proprietary software} % Role
    {Rumo Devops implementation }% Title
    {Rumo} % Location
    {Aug. 2018 - PRESENT} % Date(s)
    {
      \begin{cvitems} % Description(s)
       \item{In this project I had the opportunity to be responsible for the implementation of the DevOps methodology in Rumo with the automation of deploys of SOA, OSB and Web projects (Using Maven and Jenkins pipeline (Groovy) to deploy in the company Weblogic servers), automation of the update of the Jira boards (Using Jenkins Pipelines), orchestration of containers (Using Docker/Docker Compose and Kubernetes). }
       \item{In addition I had the opportunity to work in the changes of configuration management in the RUMO IT area as the implementation and deployment of migration from SVN Version Control to GIT (Bitbucket) and documentation of projects with Confluence.}
      \end{cvitems}
    }

  \cventry
    {https://www.activia.us.com} % Role
    {Activia Websites} % Title
    {Mirum Agency} % Location
    {Nov. 2016 - Jan. 2017} % Date(s)
    {
      \begin{cvitems} % Description(s)
        \item {Development and maintenance of Activia Danone's global websites.}
        \item {The sites were developed using the technologies Drupal 8, MySQL, JS, Node.js and others.}
      \end{cvitems}
    }

%---------------------------------------------------------

 \cventry
    {http://www.whufc.com} % Role
    {West Ham United F.C. Website} % Title
    {Mirum Agency} % Location
    {Apr. 2016 - Sep. 2016} % Date(s)
    {
      \begin{cvitems} % Description(s)
        \item {Backend Development of the website of the English Soccer Club West Ham United.}
         \item { Technologies used: Drupal 8, PHP, JS, MySQL, Sass, Node.js, Bootstrap and others.}
      \end{cvitems}
    }

%---------------------------------------------------------

 \cventry
    {http://vfleets.com.br} % Role
    {Veltec MapSrv System} % Title
    {Veltec} % Location
    {Jul. 2015 - Feb. 2016} % Date(s)
    {
      \begin{cvitems} % Description(s)
        \item {Web system based on OpenStreetMap stack for creating and processing maps for systems used by customers of the company Veltec. This system consists of several modules taking advantage of proprietary and open source projects to enabling editing and rendering maps as well other features such as geocoding a proprietary way by the company.}
         \item {In the system development was taken advantage of the best open source technologies, using, PostgreSQL, PostGIS, Ruby On Rails, Node.js, Python, used PHP, among others technologies.}
      \end{cvitems}
    }

%---------------------------------------------------------
\end{cventries}
